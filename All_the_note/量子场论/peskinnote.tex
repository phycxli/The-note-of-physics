\documentclass[a4paper]{article}
\usepackage{amsmath}
\usepackage[UTF8]{ctex}
\usepackage{tikz}
\usepackage{tikz-feynman}
\usepackage{theorem}
\usepackage{geometry}
\usepackage{simplewick}

\geometry{a4paper,left=2cm,right=2cm,top=2cm,bottom=2cm}

\newtheorem{proof}{答案}[section]

\begin{document}
\title{Peskin的答案}
\author{李成蹊\\中科院物理所}
\date{}
\maketitle
\begin{abstract}
    大家好,我叫小李哥,从小到大,我就思考什么是好书?随着年龄增长,我越来越认识到有题有答案就是好书!我是中国人,我的答案不是给外国人看的,我只想给我亲爱的中国同胞们看。希望我的中国人同胞们在物理上能更进一步。我的答案是个小学生也能看懂,无比详细。
\end{abstract}
\section{相互作用场和费曼图}
\subsection{经典源的微扰}
让我们回到经典源产生K-G粒子的问题上来。回顾第二章,这个过程可以用哈密顿
\begin{equation}
    H=H_0+\int d^3x(-j(t,\mathbf{x})\phi(x))
\end{equation}
来描述,其中$H_0$是K-G自由哈密顿,$\phi(x)$是K-G场,并且$j(x)$是$c$数标量函数。我们发现,如果系统在源打开之前处于真空态,源将产生平均数量
\begin{equation}
    \langle N\rangle=\int \frac{d^3p}{(2\pi)^3}\frac{1}{2E_{\mathbf{p}}}|\tilde{j}(p)|^2
\end{equation}
的粒子。在这个问题中,我们将通过在源的强度上使用微扰展开来验证这一说法,并提取更详细的信息。
\begin{enumerate}
    \item 证明源不产生粒子的概率为
    \begin{equation}
        P(0)=|\langle 0|T\left\{\exp[i\int d^4xj(x)\phi_I(x)]\right\}|0\rangle|^2
    \end{equation}
    \begin{proof}
        所谓源不产生粒子,意味着前后初末态,粒子数并没有发生任何变化。我们又知道,\textbf{系统在源打开之前处于真空态}。所以前后初末态,系统都在真空态。我们知道在相互作用表象里,时间演化算符为
        \begin{equation}
            U(t-t_0)=T[e^{-i\int_{t_0}^t dt' H(t')}]
        \end{equation}
        所以这里我们如果把经典源项看成微扰$H_I=\int d^3x(-j(t,\mathbf{x})\phi_I(x))$,问题就变得简单多了。我们考虑真空态$|0\rangle$从无穷过去开始演化到无穷未来,这个态演化成了
        \begin{equation}
            T[e^{-i\int_{-\infty}^{+\infty} dt H_I(t)}]|0\rangle
        \end{equation}
        剩下就简单多了,往无穷过去的初态真空投影内积即可得到概率振幅:
        \begin{equation}
            \langle 0|T[e^{-i\int_{-\infty}^{+\infty} dt H_I(t)}]|0\rangle=\langle 0|T[e^{i\int_{-\infty}^{+\infty} d^4 x j(x)\phi_I(x)}]|0\rangle
        \end{equation}
        $P(0)=|\langle 0|T[e^{i\int_{-\infty}^{+\infty} d^4 x j(x)\phi_I(x)}]|0\rangle|^2$得证
    \end{proof}
    \item 计算到$j^2$项的$P(0)$,并且证明$P(0)=1-\lambda+\mathcal{O}(j^4)$,其中$\lambda$表示$\langle N\rangle$.
    \begin{proof}
        泰勒级数$e^x=1+x+\frac{x^2}{2!}+\cdots$,得到
        \begin{equation}
            e^{i\int d^4x j(x)\phi_I(x)}=1+i\int d^4 x j(x)\phi_I(x)+\frac{1}{2!}\left(i\int d^4 xj(x)\phi_I(x)\right)\left(i\int d^4 yj(y)\phi_I(y)\right)+\cdots
        \end{equation}
        考虑到$\langle 0|\phi_I(x)|0\rangle=0$,我们有
        \begin{equation}
            \begin{split}
                \langle 0|T\exp[i\int d^4x j(x)\phi_I(x)]|0\rangle&=1-\frac{1}{2}\langle 0|T\{\int d^4xj(x)\phi_I(x)\int d^4y j(y)\phi_I(y)\}|0\rangle+\mathcal{O}(j^4)\\
                &=1-\frac{1}{2}\int d^4xd^4yj(x)j(y)\langle 0|T\{\phi_I(x)\phi_I(y)\}|0\rangle+\mathcal{O}(j^4)\\
                &=1-\frac{1}{2}\int d^4xd^4yj(x)j(y)D(x-y)+\mathcal{O}(j^4)\\
                &=1-\frac{1}{2}\int d^4xd^4yj(x)j(y)\int \frac{d^3p}{(2\pi)^3}\frac{1}{2E_{\mathbf{p}}}e^{-ip\cdot(x-y)}+\mathcal{O}(j^4)\\
                &=1-\frac{1}{2}\int \frac{d^3p}{(2\pi)^3}\frac{1}{2E_{\mathbf{p}}}|\tilde{j}(p)|^2+\mathcal{O}(j^4)
            \end{split}
        \end{equation}
        所以$\langle 0|T\exp[i\int d^4xj(x)\phi_I(x)]|0\rangle=1-\frac{\lambda}{2}+\mathcal{O}(j^4)$,即$P(0)=|\langle 0|T\exp[i\int d^4xj(x)\phi_I(x)]|0\rangle|^2=1-\lambda+\mathcal{O}(j^4)$
    \end{proof}
    \item 将上小题部分中计算的项表示为费曼图。现在用费曼图表示$P(0)$的整个微扰级数。证明这个级数是指数的,因此可以精确地求和:$P(0)=e^{-\lambda}$
    \begin{proof}
        首先我们计算$\langle 0|T\exp[i\int d^4xj(x)\phi_I(x)]|0\rangle$首次项,得到
        \begin{equation}
            \langle 0|T\exp[i\int d^4x j(x)\phi_I(x)]|0\rangle=1-\frac{\lambda}{2}
        \end{equation}
        所以我们可以得到费曼规则:
        \begin{tikzpicture}
            \begin{feynman}
                \vertex (a1) {\(x\)};
                \vertex [right=2cm of a1] (a2) {\(y\)};
                \diagram*{
                    (a1) -- [scalar] (a2);
                };
            \end{feynman}
        \end{tikzpicture}$=-\lambda$

        例如计算$j^4$项的时候我们有
        \begin{center}
            \begin{tikzpicture}
                \begin{feynman}
                    \vertex (a1) {\(x\)};
                    \vertex [right=2cm of a1] (a2) {\(y\)};
                    \vertex [below=1cm of a1] (a3) {\(z\)};
                    \vertex [right=2cm of a3] (a4) {\(w\)};
                    \diagram*{
                        (a1) -- [scalar] (a2);
                        (a3) -- [scalar] (a4);
                    };
                \end{feynman}
            \end{tikzpicture}$+$
            \begin{tikzpicture}
                \begin{feynman}
                    \vertex (a1) {\(x\)};
                    \vertex [right=1cm of a1] (a2) {\(y\)};
                    \vertex [below=2cm of a1] (a3) {\(z\)};
                    \vertex [right=1cm of a3] (a4) {\(w\)};
                    \diagram*{
                        (a1) -- [scalar] (a3);
                        (a2) -- [scalar] (a4);
                    };
                \end{feynman}
            \end{tikzpicture}$+$
            \begin{tikzpicture}
                \begin{feynman}
                    \vertex (a1) {\(x\)};
                    \vertex [right=2cm of a1] (a2) {\(y\)};
                    \vertex [below=1cm of a1] (a3) {\(z\)};
                    \vertex [right=2cm of a3] (a4) {\(w\)};
                    \diagram*{
                        (a1) -- [scalar] (a4);
                        (a2) -- [scalar] (a3);
                    };
                \end{feynman}
            \end{tikzpicture}$=3(-\lambda)^2$
        \end{center}
        同样的我们考虑第$2n$项得到
        \begin{center}
            \begin{tikzpicture}
                \draw[dashed] (-1,0)node[left]{\(x_1\)} -- (1,0)node[right]{\(x_2\)};
                \draw[dashed] (-1,-1)node[left]{\(x_3\)} -- (1,-1)node[right]{\(x_4\)};
                \node (t) at (0,-2.5) {\(\cdots\)};
                \draw[dashed] (-1,-2)node[left]{\(x_5\)} -- (1,-2)node[right]{\(x_6\)};
            \end{tikzpicture}$=(2n-1)!!(-\lambda)^n$
        \end{center}
        其中由于对称性,对称重复计算的对称因子为$(2n-1)!!$. 所以总的跃迁振幅为
        \begin{equation}
            \langle 0|T\exp[i\int d^4x j(x)\phi_I(x)]|0\rangle=1-\frac{1}{2!}\lambda+\frac{1}{4!}3\lambda+\cdots+\frac{1}{(2n)!}(2n-1)!!(-\lambda)^n=\sum_{n=0}\frac{(-\frac{\lambda}{2})^n}{n!}=e^{-\frac{\lambda}{2}}
        \end{equation}
        所以$P(0)=e^{-\lambda}$
    \end{proof}
    \item 计算源产生一个动量为k的粒子的概率。首先执行此计算到$\mathcal{O}(j)$,然后执行到所有阶,使用上式的技巧对级数求和。
    \begin{proof}
        我们考虑费曼规则
        \begin{equation}
            \begin{split}
                \contraction{\langle}{k}{|}{\phi}\langle k|\phi(x)=\langle 0|a_k\sqrt{2E_k}\int\frac{d^3p}{(2\pi)^3}\frac{1}{\sqrt{2E_p}}a_\mathbf{p}^\dagger e^{ip\cdot x}=\langle 0|e^{ik\cdot x}
            \end{split}
        \end{equation}
        所以我们有
        \begin{equation}
            \langle k|i\int d^4j(x)\phi(x)|0\rangle=i\int d^4x j(x)e^{ik\cdot x}=i\tilde{j}(p)
        \end{equation}
        \begin{equation}
            \contraction{}{\phi}{(x)}{\phi}\phi(x)\phi(y)=D_F(x-y)=\int\frac{d^4p}{(2\pi)^4}\frac{i}{p^2-m^2+i\epsilon}e^{-ip\cdot (x-y)}
        \end{equation}
        所以我们有
        \begin{equation}
            \begin{split}
                \contraction[1.5em]{i\int d^4xj(x)}{\phi}{(x)i\int d^4yj(y)}{\phi}i\int d^4xj(x)\phi(x)i\int d^4y j(y)\phi(y)&=-\int d^4xd^4y j(x)j(y)D_F(x-y)\\
                &=-\int\frac{d^3p}{(2\pi)^3}\frac{|\tilde{j}(p)|^2}{2E_\mathbf{p}}=-\lambda
            \end{split}
        \end{equation}
        考虑$\langle k|T\exp[i\int d^4x j(x)\phi_I(x)]|0\rangle$展开,由于偶数次展开项加上末态湮灭算符一共有奇数个场算符,值为$0$. 所以只存在奇数次展开式,考虑第$2n+1$次展开
        \begin{equation}
            \alpha_{2n+1}=\frac{1}{(2n+1)!}\times(2n+1)!!(-\lambda)^n\times i\tilde{j}(p)=\frac{i}{(2n)!!}(-\lambda)^n\tilde{j}(p)=\frac{i}{n!}(-\frac{\lambda}{2})^n\tilde{j}(p)
        \end{equation}
        所以$P(k)=|i\sum_{n}\frac{(-\lambda/2)^n}{n!}\tilde{j}(p)|^2=|\tilde{j}(p)|^2e^{-\lambda}$
    \end{proof}
    \item 证明产生n个粒子的概率为$P(n)=\frac{1}{n!}\lambda^n e^{-\lambda}$. 这就是Poisson分布。
    \begin{proof}
        我们考虑跃迁振幅$\langle k_1k_2\cdots k_n|T e^{i\int d^4x j(x)\phi(x)}|0\rangle$. 我们考虑对上式微扰展开,考虑第$n$阶微扰项(从这一项开始,微扰非$0$).
        \begin{equation}
            \begin{split}
                \alpha_n&=\frac{1}{n!}\langle k_1k_2\cdots k_n|T\left\{\left(i\int d^4x_1 j(x_1)\phi(x_1)\right)\cdots\left(i\int d^4x_n j(x_n)\phi(x_n)\right)\right\}|0\rangle\\
                &=\frac{i^n}{n!}
            \end{split}
        \end{equation}
    \end{proof}
\end{enumerate}

\end{document}