\documentclass{article}
\usepackage{amsmath}
\usepackage[UTF8]{ctex}
\newcommand{\mC}{\mathcal{C}}
\begin{document}
\title{非厄米对称性分类}
\author{李成蹊}
\date{}
\maketitle
\section{对称性}
对于厄米系统而言,内禀对称性属于AZ对称类:时间反演对称(TRS)、粒子-空穴对称(PHS)以及手征对称(CS)都是幺正的。这些对称性得到了厄米系统拓扑绝缘体和拓扑超流体的10重对称类。另一方面,即使存在非厄米性,AZ对称性是否完全描述了所有的内部对称性也不是无关紧要的。事实上PHS是用转置来定义的并且不能再用复共轭来描述非厄米的BdG哈密顿,因为复数共轭和置换有区别。相应地,CS与子格对称(SLS)并不一致,尽管它们在厄米性存在时是等价的。因此,对称类的总数如下文所示为38个,每个对称类都描述了内蕴的非厄米拓扑相和非厄米随机矩阵。

\subsection{对称性的分化与统一}
在我们详细描述38种对称性之前,我们总结一下非厄米物理中对称性质的变化。事实上,非厄米性以一种基本的方式分化和统一了对称性。首先为了看到对称性分化,我们考虑PHS作为一个例子。对于厄米系统,PHS被定义为
\begin{equation}
    \mC H^*\mC=-H
\end{equation}
\end{document}