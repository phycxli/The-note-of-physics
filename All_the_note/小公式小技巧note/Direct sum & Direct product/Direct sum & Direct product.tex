\documentclass[a4paper,11pt]{article}
\usepackage[UTF8]{ctex}
\usepackage{amsmath}
\usepackage{geometry}
\usepackage{amsthm}
\geometry{left=2.0cm,right=2.0cm,top=2.5cm,bottom=2.5cm}
\begin{document}
    \title{科普文之直和与直乘}
    \author{李成蹊\\Institute of Physics, Chinese Academy of Sciences}
    \date{}
    \maketitle
    \abstract{最近计算狄拉克方程、泡利矩阵、分析乱七八糟拓扑分类搞得的心烦,遂在此把相关矩阵运算公式整理一下。虽然这都是大一的内容,但舍己为人是吾辈之使命。}
    \section{直和}
    给定两个向量空间$V,W$,维数分别是$n,m$。$V$里有一组向量基底$\left\{\vec{e}_{1}, \vec{e}_{2}, \cdots, \vec{e}_{n}\right\}$,同样$W$里也有一组基底$\left\{\vec{f}_{1}, \vec{f}_{2}, \cdots, \vec{f}_{m}\right\}$.我们现在定义$n+m$个向量构成基组$\left\{\vec{e}_{1}, \cdots, \vec{e}_{n}, \vec{f}_{1}, \cdots, \vec{f}_{m}\right\}$. 由这些基组生成的线性空间叫做$V$与$W$生成的直和空间$V\oplus W$,线性空间维数为$n+m$.\\
    任意向量$\vec{v} \in V$都可以写成基底线性叠加的形式$\vec{v}=v_{1} \vec{e}_{1}+v_{2} \vec{e}_{2}+\cdots+v_{n} \vec{e}_{n}$. 线性代数里我们可以用这些叠加系数构成的列向量来代表这个向量。
    \begin{equation}
        \left.\vec{v}=\left(\begin{array}{c}
            v_{1} \\
            v_{2} \\
            \vdots \\
            v_{n}
            \end{array}\right)\right\} n
    \end{equation}
    值得注意,这个列向量是线性空间$V$的一个元素,列向量有$n$个分量。特别地,向量基底可以写成
    \begin{equation}
        \vec{e}_{1}=\left(\begin{array}{c}
            1 \\
            0 \\
            \vdots \\
            0
            \end{array}\right), \quad \vec{e}_{2}=\left(\begin{array}{c}
            0 \\
            1 \\
            \vdots \\
            0
            \end{array}\right)_{V}, \quad \cdots, \quad \vec{e}_{n}=\left(\begin{array}{c}
            0 \\
            0 \\
            \vdots \\
            1
            \end{array}\right)_{V}
    \end{equation}
    若基组是正交组,$\vec{e}_{i} \cdot \vec{e}_{j}=\delta_{i j}$,则有$v_i=\vec{e}_i\cdot\vec{v}$.同样对于$w$中的任意向量也可以有
    \begin{equation}
        \left.\vec{w}=\left(\begin{array}{c}
            w_{1} \\
            w_{2} \\
            \vdots \\
            w_{n}
            \end{array}\right)\right\} m
    \end{equation}
    基向量为
    \begin{equation}
        \vec{f}_{1}=\left(\begin{array}{c}
            1 \\
            0 \\
            \vdots \\
            0
            \end{array}\right)_{W}, \quad \vec{f}_{2}=\left(\begin{array}{c}
            0 \\
            1 \\
            \vdots \\
            0
            \end{array}\right)_{W}, \quad \cdots, \quad \vec{f}_{m}=\left(\begin{array}{c}
            0 \\
            0 \\
            \vdots \\
            1
            \end{array}\right)_{W}
    \end{equation}
    自然地,对于元素$\vec{v}\in V$和$\vec{w}\in W$在直和空间的列向量形式可以推广成
    \begin{equation}
        \left.\left.\vec{v}=\left(\begin{array}{c}
            v_{1} \\
            \vdots \\
            v_{n} \\
            \hline 0 \\
            \vdots \\
            0
            \end{array}\right)\right\} n+m, \quad \vec{w}=\left(\begin{array}{c}
            0 \\
            \vdots \\
            0 \\
            w_{1} \\
            \vdots \\
            w_{m}
            \end{array}\right)\right\} n+m
    \end{equation}
    可以看到,只是把其余部分用0填充。\\
    我们也可以定义两个非零向量的直和
    \begin{equation}
        \left.\vec{v} \oplus \vec{w}=\left(\begin{array}{c}
            v_{1} \\
            \vdots \\
            v_{n} \\
            \hline w_{1} \\
            \vdots \\
            w_{m}
            \end{array}\right)=\left(\frac{\vec{v}}{\vec{w}}\right)\right\} n+m
    \end{equation}
    数学上矩阵表示一个线性空间到另一个线性空间的映射。这里我们只讨论一个$V\mapsto V$的特殊情况,因为量子力学里大部分情况便是如此。力学量$A:V\mapsto V$
    \begin{equation}
        \vec{v}\mapsto A\vec{v}
    \end{equation}
    \begin{equation}
        \left(\begin{array}{c}
            v_{1} \\
            v_{2} \\
            \vdots \\
            v_{n}
            \end{array}\right) \longmapsto\left(\begin{array}{cccc}
            a_{11} & a_{12} & \cdots & a_{1 n} \\
            a_{22} & a_{22} & \cdots & a_{2 n} \\
            \vdots & \vdots & \ddots & \vdots \\
            a_{n 1} & a_{n 2} & \cdots & a_{n n}
            \end{array}\right)_{V}\left(\begin{array}{c}
            v_{1} \\
            v_{2} \\
            \vdots \\
            v_{n}
            \end{array}\right)_{V}
    \end{equation}
    类似的有矩阵$B:W\mapsto W,\vec{w}\mapsto B\vec{w}$\\
    在直和空间里,我们希望矩阵$A,B$经过某种运算后作用在向量上保持$\vec{v}\mapsto A\vec{v}$和$\vec{w}\mapsto B\vec{w}$。这样的运算记成$A\oplus B$. 运算结果可以利用分块对角矩阵构造
    \begin{equation}
        A \oplus B=\left(\begin{array}{c|c}
            A & 0_{n \times m} \\
            \hline 0_{m \times n} & B
            \end{array}\right)
    \end{equation}
    例如$n=2,m=3$的情况
    \begin{equation}
        A=\left(\begin{array}{cc}
            a_{11} & a_{12} \\
            a_{21} & a_{22}
            \end{array}\right)_{V}, \quad B=\left(\begin{array}{ccc}
            b_{11} & b_{12} & b_{13} \\
            b_{21} & b_{22} & b_{23} \\
            b_{31} & b_{32} & b_{33}
            \end{array}\right)_{W}
    \end{equation}
    他们的直和为
    \begin{equation}
        A \oplus B=\left(\begin{array}{cc|ccc}
            a_{11} & a_{12} & 0 & 0 & 0 \\
            a_{21} & a_{22} & 0 & 0 & 0 \\
            \hline 0 & 0 & b_{11} & b_{12} & b_{13} \\
            0 & 0 & b_{21} & b_{22} & b_{23} \\
            0 & 0 & b_{31} & b_{32} & b_{33}
            \end{array}\right)
    \end{equation}
    显然$A,B$分别作用在不同的子空间里。显然直和矩阵$A\oplus B$作用在直和向量$\vec{v} \oplus \vec{w}$为
    \begin{equation}
        (A \oplus B)(\vec{v} \oplus \vec{w})=\left(\begin{array}{c|c}
            A & 0_{n \times m} \\
            \hline 0_{m \times n} & B
            \end{array}\right)\left(\frac{\vec{v}}{\vec{w}}\right)=\left(\frac{A \vec{v}}{B \vec{w}}\right)=(A \vec{v}) \oplus(B \vec{w})
    \end{equation}
    若有两个直和矩阵相乘,利用对角矩阵运算显然得到
    \begin{equation}
        \left(A_{1} \oplus B_{1}\right)\left(A_{2} \oplus B_{2}\right)=\left(A_{1} A_{2}\right) \oplus\left(B_{1} B_{2}\right)
    \end{equation}
    直和空间$V\oplus W$一共有$(n+m)^2$个线性无关矩阵,而子空间$V$和$W$分别只有$n^2,m^2$个线性无关矩阵,这说明并不是所有在直和空间中的矩阵都能表示成直和形式。\\
    利用分块矩阵行列式规则显然得到
    \begin{equation}
        \operatorname{det}(A \oplus B)=(\operatorname{det} A)(\operatorname{det} B)
    \end{equation}
    分块矩阵求迹得到的是各个子块的迹之和
    \begin{equation}
        \operatorname{Tr}(A \oplus B)=(\operatorname{Tr} A)+(\operatorname{Tr} B)
    \end{equation}
    \section{直积}
    直积这个词在有些书上叫张量积、Kronecker积

    和之前一样,给定两个线性空间$V,W$,分别$n,m$维. $V$中有一组基底$\left\{\vec{e}_{1}, \vec{e}_{2}, \cdots, \vec{e}_{n}\right\}$,$W$中也有一组基底$\left\{\vec{f}_{1}, \vec{f}_{2}, \cdots, \vec{f}_{m}\right\}$.我们现在定义$nm$个基向量$\vec{e}_{i} \otimes \vec{f}_{j}$,其中$i=1,\cdots,n$和$j=1,\cdots,m$.$\otimes$具体运算下面将会讲明,可以看到$\vec{e}_i\otimes\vec{e}_j$有两个指标,这也是为什么直积也叫张量积的原因。

    张量积是双线性的,即在$V$中是线性的,在$W$中也是线性的. 如果有多于两个线性空间,则叫多重线性的。这意味着$\vec{v} \otimes \vec{w}=\left(\sum_{i}^{n} v_{i} \vec{e}_{i}\right) \otimes\left(\sum_{j}^{m} w_{j} \vec{f}_{j}\right)=\sum_{i}^{n} \sum_{j}^{m} v_{i} w_{j}\left(\vec{e}_{i} \otimes \vec{f}_{j}\right)$. 换句话说由向量基底$\vec{e}_i\otimes\vec{f}_i$生成的线性空间$V\otimes W$中的向量由这组基$\vec{v}_i\otimes\vec{w}_j$线性表出,系数为$v_iw_j$

    一种使它非常明确的方法是把它写出来。例如假设$n=2,m=3$。张量积空间的维数为$nm=6$新的基底分别是$\vec{e}_{1} \otimes \vec{f}_{1}, \vec{e}_{1} \otimes \vec{f}_{2}, \vec{e}_{1} \otimes \vec{f}_{3}, \vec{e}_{2} \otimes \vec{f}_{1}, \vec{e}_{2} \otimes \vec{f}_{2}, \vec{e}_{2} \otimes \vec{f}_{3}$. 写出这6个基底对应的列向量
    \begin{equation}
        \begin{split}
            &\vec{e}_{1} \otimes \vec{f}_{1}=\left(\begin{array}{c}
                1 \\
                0 \\
                0 \\
                \hline 0 \\
                0 \\
                0
                \end{array}\right), \quad \vec{e}_{1} \otimes \vec{f}_{2}=\left(\begin{array}{c}
                0 \\
                1 \\
                0 \\
                0 \\
                0 \\
                0
                \end{array}\right), \quad \vec{e}_{1} \otimes \vec{f}_{3}=\left(\begin{array}{c}
                0 \\
                0 \\
                1 \\
                \hline 0 \\
                0 \\
                0
                \end{array}\right)\\
                &\vec{e}_{2} \otimes \vec{f}_{1}=\left(\begin{array}{c}
                    0 \\
                    0 \\
                    0 \\
                    \hline 1 \\
                    0 \\
                    0
                    \end{array}\right), \quad \vec{e}_{2} \otimes \vec{f}_{2}=\left(\begin{array}{c}
                    0 \\
                    0 \\
                    0 \\
                    \hline 0 \\
                    1 \\
                    0
                    \end{array}\right), \quad \vec{e}_{2} \otimes \vec{f}_{3}=\left(\begin{array}{c}
                    0 \\
                    0 \\
                    0 \\
                    \hline 0 \\
                    0 \\
                    1
                    \end{array}\right)
        \end{split}
    \end{equation}
    这里横线把基向量分割成了两个不同的子集,第一部分由$\vec{e}_1$生成,第二部分由$\vec{e}_2$生成。对于一般的向量$\vec{v}\in V,\vec{w}\in W$,张量积为
    \begin{equation}
        \vec{v} \otimes \vec{w}=\left(\begin{array}{l}
            v_{1} w_{1} \\
            v_{1} w_{2} \\
            v_{1} w_{3} \\
            \hline v_{2} w_{1} \\
            v_{2} w_{2} \\
            v_{2} w_{3}
            \end{array}\right)
    \end{equation}
    在数学上矩阵是一个线性空间到另一个线性空间的线性映射。同样的我们只关注量子力学感兴趣的$A:V\mapsto V$
    \begin{equation}
        \vec{v} \longmapsto A \vec{v}
    \end{equation}
    或者说
    \begin{equation}
        \left(\begin{array}{c}
            v_{1} \\
            v_{2} \\
            \vdots \\
            v_{n}
            \end{array}\right) \longmapsto\left(\begin{array}{cccc}
            a_{11} & a_{12} & \cdots & a_{1 n} \\
            a_{22} & a_{22} & \cdots & a_{2 n} \\
            \vdots & \vdots & \ddots & \vdots \\
            a_{n 1} & a_{n 2} & \cdots & a_{n n}
            \end{array}\right)\left(\begin{array}{c}
            v_{1} \\
            v_{2} \\
            \vdots \\
            v_{n}
            \end{array}\right)_{V}
    \end{equation}
    在张量积空间上,相同的矩阵仍然可以作用在矢量上,使得$\vec{v} \mapsto A \vec{v}$但$\vec{w} \mapsto \vec{w}$保持不变。这个矩阵可以写成$A\otimes I$,其中$I$是恒等矩阵。在之前的$n=2,m=3$例子里,矩阵$A$是$2\times2$维矩阵,显然$A\otimes I$是$6\times6$维矩阵
    \begin{equation}
        A \otimes I=\left(\begin{array}{ccc|ccc}
            a_{11} & 0 & 0 & a_{12} & 0 & 0 \\
            0 & a_{11} & 0 & 0 & a_{12} & 0 \\
            0 & 0 & a_{11} & 0 & 0 & a_{12} \\
            \hline a_{21} & 0 & 0 & a_{22} & 0 & 0 \\
            0 & a_{21} & 0 & 0 & a_{22} & 0 \\
            0 & 0 & a_{21} & 0 & 0 & a_{22}
            \end{array}\right)
    \end{equation}
    把这个矩阵作用在$\vec{v}\otimes\vec{w}$上就知道了之所以这样构造的原因
    \begin{equation}
        \begin{aligned}
            (A \otimes I)(\vec{v} \otimes \vec{w}) &=\left(\begin{array}{ccc|ccc}
            a_{11} & 0 & 0 & a_{12} & 0 & 0 \\
            0 & a_{11} & 0 & 0 & a_{12} & 0 \\
            0 & 0 & a_{11} & 0 & 0 & a_{12} \\
            \hline a_{21} & 0 & 0 & a_{22} & 0 & 0 \\
            0 & a_{21} & 0 & 0 & a_{22} & 0 \\
            0 & 0 & a_{21} & 0 & 0 & a_{22}
            \end{array}\right)\left(\begin{array}{c}
            v_{1} w_{1} \\
            v_{1} w_{2} \\
            v_{1} w_{3} \\
            \hline v_{2} w_{1} \\
            v_{2} w_{2} \\
            v_{2} w_{3}
            \end{array}\right) \\
            =&\left(\begin{array}{c}
            \left(a_{11} v_{1}+a_{12} v_{2}\right) w_{1} \\
            \left(a_{11} v_{1}+a_{12} v_{2}\right) w_{2} \\
            \left(a_{11} v_{1}+a_{12} v_{2}\right) w_{3} \\
            \hline\left(a_{21} v_{1}+a_{22} v_{2}\right) w_{1} \\
            \left(a_{21} v_{1}+a_{22} v_{2}\right) w_{2} \\
            \left(a_{21} v_{1}+a_{22} v_{2}\right) w_{3}
            \end{array}\right) &=(A \vec{v}) \otimes \vec{w}
            \end{aligned}
    \end{equation}
    显然矩阵$A$仅仅只作用在$\vec{v}\in V$上并且保持$\vec{w}\in W$不变。

    类似的对于矩阵$B:W\mapsto W$, $\vec{w}\mapsto B\vec{w}$. 同样的也能用$I\otimes B$作用在$V\otimes W$中的矢量。其中
    \begin{equation}
        I \otimes B=\left(\begin{array}{ccc|ccc}
            b_{11} & b_{12} & b_{13} & 0 & 0 & 0 \\
            b_{21} & b_{22} & b_{23} & 0 & 0 & 0 \\
            b_{31} & b_{32} & b_{33} & 0 & 0 & 0 \\
            \hline 0 & 0 & 0 & b_{11} & b_{12} & b_{13} \\
            0 & 0 & 0 & b_{21} & b_{22} & b_{23} \\
            0 & 0 & 0 & b_{31} & b_{32} & b_{33}
            \end{array}\right)
    \end{equation}
    作用在$\vec{v}\otimes\vec{w}$为
    \begin{equation}
        \begin{aligned}
            (I \otimes B)(\vec{v} \otimes \vec{w}) &=\left(\begin{array}{ccc|ccc}
            b_{11} & b_{12} & b_{13} & 0 & 0 & 0 \\
            b_{21} & b_{22} & b_{23} & 0 & 0 & 0 \\
            b_{31} & b_{32} & b_{33} & 0 & 0 & 0 \\
            \hline 0 & 0 & 0 & b_{11} & b_{12} & b_{13} \\
            0 & 0 & 0 & b_{21} & b_{22} & b_{23} \\
            0 & 0 & 0 & b_{31} & b_{32} & b_{33}
            \end{array}\right)\left(\begin{array}{c}
            v_{1} w_{1} \\
            v_{1} w_{2} \\
            v_{1} w_{3} \\
            \hline v_{2} w_{1} \\
            v_{2} w_{2} \\
            v_{2} w_{3}
            \end{array}\right) \\
            =&\left(\begin{array}{c}
            v_{1}\left(b_{11} w_{1}+b_{12} w_{2}+b_{13} w_{3}\right) \\
            v_{1}\left(b_{21} w_{1}+b_{22} w_{2}+b_{23} w_{3}\right) \\
            v_{1}\left(b_{31} w_{1}+b_{32} w_{2}+b_{33} w_{3}\right) \\
            \hline v_{2}\left(b_{11} w_{1}+b_{12} w_{2}+b_{13} w_{3}\right) \\
            v_{2}\left(b_{21} w_{1}+b_{22} w_{2}+b_{23} w_{3}\right) \\
            v_{2}\left(b_{31} w_{1}+b_{32} w_{2}+b_{33} w_{3}\right)
            \end{array}\right) &=\vec{v} \otimes(B \vec{w})
            \end{aligned}
    \end{equation}
    一般而言
    \begin{equation}
        \begin{split}
            (A \otimes I)(\vec{v} \otimes \vec{w})&=(A \vec{v}) \otimes \vec{w}\\
            (I \otimes B)(\vec{v} \otimes \vec{w})&=\vec{v} \otimes(B \vec{w})
        \end{split}
    \end{equation}
    利用上面式子很容易证明下列式子
    \begin{equation}
        \begin{array}{l}
            \left(A_{1} \otimes I\right)\left(A_{2} \otimes I\right)=\left(A_{1} A_{2}\right) \otimes I \\
            \left(I \otimes B_{1}\right)\left(I \otimes B_{2}\right)=I \otimes\left(B_{1} B_{2}\right) \\
            \left(A_{1} \otimes B_{1}\right)\left(A_{2} \otimes B_{2}\right)=\left(A_{1}A_{2}\right) \otimes\left(B_{1} B_{2}\right) \\
            (A \otimes I)(I \otimes B)=(I \otimes B)(A \otimes I)=(A \otimes B)
            \end{array}
    \end{equation}
    \begin{proof}
        \begin{equation}
            (A_1\otimes I)(A_2\otimes I)(\vec{v}\otimes\vec{w})=(A_1\otimes I)(A_2\vec{v}\otimes\vec{w})=(A_1A_2\vec{v}\otimes\vec{w})=((A_1 A_2)\otimes I)(\vec{v}\otimes\vec{w})
        \end{equation}
        第二三式类似证明,不再赘述
        \begin{equation}
            (A\otimes I)(I\otimes B)(\vec{v}\otimes\vec{w})=(A\otimes I)(v\otimes B\vec{w})=(A\vec{v})\otimes(B\vec{w})=(A\otimes B)(\vec{v}\otimes\vec{w})=(I \otimes B)(A \otimes I)(\vec{v}\otimes\vec{w})
        \end{equation}
    \end{proof}
    最后一式定义了$A\otimes B$,也可以写出矩阵形式
    \begin{equation}
        \begin{split}
        A \otimes B&=(A\otimes I)(I\otimes B)\\
        &=\left(\begin{array}{ccc|ccc}
            a_{11} & 0 & 0 & a_{12} & 0 & 0 \\
            0 & a_{11} & 0 & 0 & a_{12} & 0 \\
            0 & 0 & a_{11} & 0 & 0 & a_{12} \\
            \hline a_{21} & 0 & 0 & a_{22} & 0 & 0 \\
            0 & a_{21} & 0 & 0 & a_{22} & 0 \\
            0 & 0 & a_{21} & 0 & 0 & a_{22}
            \end{array}\right)\left(\begin{array}{ccc|ccc}
                b_{11} & b_{12} & b_{13} & 0 & 0 & 0 \\
                b_{21} & b_{22} & b_{23} & 0 & 0 & 0 \\
                b_{31} & b_{32} & b_{33} & 0 & 0 & 0 \\
                \hline 0 & 0 & 0 & b_{11} & b_{12} & b_{13} \\
                0 & 0 & 0 & b_{21} & b_{22} & b_{23} \\
                0 & 0 & 0 & b_{31} & b_{32} & b_{33}
                \end{array}\right)\\
                &=\left(\begin{array}{ccc|ccc}
            a_{11} b_{11} & a_{11} b_{12} & a_{11} b_{13} & a_{12} b_{11} & a_{12} b_{12} & a_{12} b_{13} \\
            a_{11} b_{21} & a_{11} b_{22} & a_{11} b_{23} & a_{12} b_{21} & a_{12} b_{22} & a_{12} b_{23} \\
            a_{11} b_{31} & a_{11} b_{32} & a_{11} b_{33} & a_{12} b_{31} & a_{12} b_{32} & a_{12} b_{33} \\
            \hline a_{21} b_{11} & a_{21} b_{12} & a_{21} b_{13} & a_{22} b_{11} & a_{22} b_{12} & a_{22} b_{13} \\
            a_{21} b_{21} & a_{21} b_{22} & a_{21} b_{23} & a_{22} b_{21} & a_{22} b_{22} & a_{22} b_{23} \\
            a_{21} b_{31} & a_{21} b_{32} & a_{21} b_{33} & a_{22} b_{31} & a_{22} b_{32} & a_{22} b_{33}
            \end{array}\right)
        \end{split}
    \end{equation}
值得注意,张量积空间中不是所有矩阵都能写成两个子空间矩阵张量积的形式。此外还有两个常用公式
\begin{equation}
    \begin{array}{l}
        \operatorname{det}(A \otimes B)=(\operatorname{det} A)^{m}(\operatorname{det} B)^{n} \\
        \operatorname{Tr}(A \otimes B)=(\operatorname{Tr} A)(\operatorname{Tr} B)
        \end{array}
\end{equation}
\end{document}