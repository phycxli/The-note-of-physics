\documentclass[a4paper,11pt]{article}
\usepackage[UTF8]{ctex}
\usepackage{amsmath}
\usepackage{geometry}
\geometry{left=1.5cm,right=1.5cm,top=2.5cm,bottom=2.5cm}

\newtheorem{definition}{\hspace{2em}定义}[section]
\newtheorem{theorem}{\hspace{2em}定理}[section]
\newtheorem{lemma}{\hspace{2em}引理}[section]
\newtheorem{proof}{证明}[section]
\newtheorem{example}{例}[section]
\newtheorem{corollary}{推论}[section]
\newtheorem{hypothesis}{假设}[section]
\newtheorem{property}{性质}[section]
\newtheorem{problem}{问题}[section]
\newtheorem{formula}{公式}[section]

\begin{document}
    \title{复数矩阵的对角化方法——Jordan标准型}
    \author{李成蹊\\中科院物理所}
    \maketitle
    \abstract{我们知道,如果矩阵能相似与对角矩阵,肯定会给理论和实际应用带来不少方便。可惜的是,不是所有方阵都能相似与对角矩阵。对于复数矩阵在实数域上甚至不能实对角化,而在物理上我们经常遇到复数矩阵对角化问题。}
    \section{多项式矩阵、$\lambda-$矩阵}
    \begin{definition}
        如果矩阵中每个元素都是$\lambda$的多项式,则称该矩阵为$\lambda-$矩阵
    \end{definition}
    例如
    \begin{equation*}
        A(\lambda)=\begin{bmatrix}
            2\lambda^2-3\lambda&\lambda-2&0\\
            \lambda^2-\lambda+1&2&\lambda^3
        \end{bmatrix}
    \end{equation*}
    这是一个$2\times 3$的$\lambda-$矩阵.

    元素全是数的矩阵叫做数元矩阵,数元矩阵可以看成特殊的多项式矩阵。
    \begin{definition}
        定义多项式矩阵下列三种初等变换
        \begin{enumerate}
            \item 非零数$k$乘$A(\lambda)$的某行(列);
            \item 将$A(\lambda)$的某行(列)的$g(\lambda)$倍加到另一行(列),其中$g(\lambda)$是$\lambda$的多项式;
            \item 互换$A(\lambda)$的两行(列)。
        \end{enumerate}
    \end{definition}
    类似于数元矩阵的倍法变换、消法变换、换法变换。这里要注意两点:倍法变换的$k$必须是非零常数,而不能是其他;消法变换的$g(\lambda)$是多项式,不是能是分式。
    \begin{definition}
        设$A(\lambda)$金额$B(\lambda)$是两个同型的多项式矩阵,如果$A(\lambda)$可以经过有限次初等变换到$B(\lambda)$,则说$A(\lambda)$与$B(\lambda)$等价,记作$A(\lambda)\cong B(\lambda)$
    \end{definition}
    对于$n$阶数元矩阵$A$特征矩阵$\lambda E-A$是一个特定的多项式矩阵。这里不加证明给出如下结果
    \begin{theorem}
        对于$n$阶数元矩阵$A$,总有
        \begin{equation*}
            \lambda E-A\cong G(\lambda)=\begin{bmatrix}
                g_1(\lambda)&\quad&\quad&\quad\\
                \quad&g_2(\lambda)&\quad&\quad\\
                \quad&\quad&\cdots&\quad\\
                \quad&\quad&\quad&g_n(\lambda)
            \end{bmatrix}
        \end{equation*}
        其中$g_1(\lambda),g_2(\lambda),\cdots,g_n(\lambda)$都是首项系数为$1$的多项式,并且
        \begin{equation*}
            |\lambda E-A|=g_1(\lambda)g_2(\lambda)\cdots g_n(\lambda)
        \end{equation*}
    \end{theorem}
    $\lambda E-A$经过有限次初等变换得到$G(\lambda)$,根据初等变换对矩阵相应行列式值的影响,可知$G(\lambda)$与$|\lambda E-A|$最多相差非零常数倍。注意到$|G(\lambda)|$与$|\lambda E-A|$的值表达式都是首项系数为$1$的多项式,所以上式成立。
    \begin{definition}
        对于$n$阶数元矩阵$A$,设$\lambda E-A$经初等变换化为对角矩阵$G(\lambda)$。将$g_1(\lambda),g_2(\lambda),\cdots,g_n(\lambda)$中的每个非常数多项式做复数域上的标准分解,各个分解式中的每一个一次因式方幂称为$A$的一个初等因子,初等因子的全体称为$A$的初等因子组。
    \end{definition}
    例如对于$5$阶数元矩阵$A$
    \begin{equation*}
        \lambda E-A\cong\begin{bmatrix}
            1&\quad&\quad&\quad&\quad\\
            \quad&1&\quad&\quad&\quad\\
            \quad&\quad&\lambda(\lambda-1)&\quad&\quad\\
            \quad&\quad&\quad&1&\quad\\
            \quad&\quad&\quad&\quad&\lambda(\lambda-1)^2
        \end{bmatrix}
    \end{equation*}
    $A$的初等因子组为:$\lambda,\lambda-1,\lambda,(\lambda-1)^2$
    注意三点
    \begin{enumerate}
        \item 方阵$A$的所有初等因子的乘积就是$A$的特征多项式
        \item 每个初等因子都和矩阵$A$的某个特征值相应,即如果$(\lambda-\lambda_i)^{m_i}$是$A$的一个初等因子,则$\lambda_i$一定是$A$的一个特征值
        \item $n$阶方阵$A$的所有初等因子幂次之和为$n$
    \end{enumerate}
    方阵$A$与某一个特征值对应的初等因子未必只有一个,所以一般不能从$A$的特征多形式的标准分解式$\psi(\lambda)=(\lambda-\lambda_1)^{n_1}(\lambda-\lambda_2)^{n_2}\cdots(\lambda-\lambda_t)^{n_t}$直接得到初等因子组为$(\lambda-\lambda_1)^{n_1},\cdots,(\lambda-\lambda_t)^{n_t}$.

    在不计各初等因子相互次序的意义下,给定方阵$A$的初等因子组是唯一的,不会因为$\lambda E-A$所化成的对角矩阵不同而有所改变。
    \section{矩阵的Jordan标准型}
    \begin{theorem}
        在复数域上,如果$n$阶矩阵$A$的全部初等因子为
        \begin{equation*}
            (\lambda-\lambda_1)^{m_1},(\lambda-\lambda_2)^{m_2},\cdots,(\lambda-\lambda_s)^{m_s}
        \end{equation*}
        则有
        \begin{equation*}
            A\sim J=\begin{bmatrix}
                J_1&\quad&\quad&\quad\\
                \quad&J_2&\quad&\quad\\
                \quad&\quad&\cdots&\quad\\
                \quad&\quad&\quad&J_s
            \end{bmatrix}
        \end{equation*}
        其中
        \begin{equation*}
            J_i=\begin{bmatrix}
                \lambda_i&1&\quad&\quad&\quad\\
                \quad&\lambda_i&1&\quad&\quad\\
                \quad&\quad&\cdots&\cdots&\quad\\
                \quad&\quad&\quad&\cdots&1\\
                \quad&\quad&\quad&\quad&\lambda_i
            \end{bmatrix}_{m_i\times m_i},i=1,2,\cdots,s
        \end{equation*}
    \end{theorem}
    该定理中的分块对角矩阵$J$称为矩阵$A$的Jordan标准型,称为Jordan型。Jordan型中各个子块$J_i$叫做Jordan块。每个Jordan块$J_i$恰好与$A$的一个初等因子$(\lambda-\lambda_i)^{m_i}$相对应。

    例如之前的$5$阶数元矩阵,初等因子组为$\lambda,\lambda-1,\lambda,(\lambda-1)^2$,对应的Jordan块
    \begin{equation*}
        J_1=(0),J_2=(1),J_3=(0),J_4=\begin{bmatrix}
            1&1\\
            0&1
        \end{bmatrix}
    \end{equation*}
    所以对应的Jordan标准型为
    \begin{equation*}
        J=\begin{bmatrix}
            0&\quad&\quad&\quad&\quad\\
            \quad&1&\quad&\quad&\quad\\
            \quad&\quad&0&\quad&\quad\\
            \quad&\quad&\quad&1&1\\
            \quad&\quad&\quad&\quad&1
        \end{bmatrix}
    \end{equation*}
    \begin{enumerate}
        \item 对于给定的方阵$A$,在不计各Jordan块排列次序的意义下,$A$的Jordan标准型是唯一的;
        \item 方阵$A$的Jordan标准型$J$是上三角矩阵,主对角线上元素恰好是$A$的全部特征值
        \item 对角矩阵本身是Jordan型,每一个主对角元都是一阶Jordan块。
    \end{enumerate}
    \begin{theorem}
        两个同阶方阵相似的充要条件是他们的Jordan形一致(不计Jordan块次序).
    \end{theorem}
    \begin{theorem}
        矩阵$A$能与对角矩阵相似的充分必要条件是它的初等因子全为一次式。
    \end{theorem}
    利用Jordan标准型可以得到下列推论
    \begin{theorem}
        如果$n$阶矩阵$A$的全部特征值是$\lambda_1,\lambda_2,\cdots,\lambda_n$,则矩阵$A^m$的全部特征值恰是$\lambda_1^m,\lambda_2^m,\cdots,\lambda_n^m$.(这里$\lambda_1,\lambda_2,\cdots,\lambda_n$可以有一些相同的数)
    \end{theorem}
    \begin{proof}
        设$A$的Jordan标准形为
        \begin{equation*}
            J=\begin{bmatrix}
                J_1&\quad&\quad&\quad\\
                \quad&J_2&\quad&\quad\\
                \quad&\quad&\cdots&\quad\\
                \quad&\quad&\quad&\quad&J_s
            \end{bmatrix}=\begin{bmatrix}
                \lambda_1&*&\quad&\quad&\quad\\
                \quad&\lambda_2&*&\quad&\quad\\
                \quad&\quad&\cdots&\cdots&\quad\\
                \quad&\quad&\quad&\cdots&*\\
                \quad&\quad&\quad&\quad&\lambda_n
            \end{bmatrix}
        \end{equation*}
        $A\sim J\Rightarrow A^m\sim J^m$. 利用上三角矩阵计算得到
        \begin{equation*}
            \begin{bmatrix}
                \lambda_1^m&*&*&\cdots&*\\
                \quad&\lambda_2^m&*&\cdots&*\\
                \quad&\quad&\cdots&\cdots&\cdots\\
                \quad&\quad&\quad&\cdots&*\\
                \quad&\quad&\quad&\quad&\lambda_n^m
            \end{bmatrix}
        \end{equation*}
        $J^m$是上三角矩阵,全部特征值就是对角元$\lambda_1,\lambda_2,\cdots,\lambda_n$. 这也就是$A^m$的全部特征值.
    \end{proof}
    \begin{theorem}(Frobeius定理)
        设$n$阶矩阵$A$全部特征值为$\lambda_1,\lambda_2,\cdots,\lambda_n$,则对任意多项式$f(\lambda)$,矩阵$f(A)$的全部特征值恰是$f(\lambda_1),f(\lambda_2),\cdots,f(\lambda_n)$
    \end{theorem}
\end{document} 