\documentclass{article}
\usepackage{amsmath}
\usepackage[UTF8]{ctex}

\begin{document}
\title{non-Hermite quantum mechanics}
\author{lcx}
\maketitle
\section{复哈密顿以及伴随}
在开始,通常使用不正交的基函数进行量子力学分析,我们将首先回顾有限维中一般复哈密顿量本征态的基本性质。设$\hat{K}=\hat{H}-\mathrm{i} \hat{\Gamma}$是一个复哈密顿,本征值态和本征值分别为$\left\{\left|\phi_{n}\right\rangle\right\} $,$\left\{\kappa_{n}\right\}$,其中$\hat{H}^{\dagger}=\hat{H}$以及$\hat{\Gamma}^{\dagger}=\hat{\Gamma}$。
\begin{equation}\label{1}
  \hat{K}|\phi_{n}\rangle=\kappa_{n}|\phi_{n}\rangle\quad \langle\phi_{n}|\hat{K}^{\dagger}=\bar{\kappa}_{n}\langle\phi_{n}|
\end{equation}
我们假设本征值$\{\kappa_n\}$并不简并. 除了$\hat{K}$的本征值之外, 引入厄米共轭矩阵$\hat{K}^\dag$的本征态$\{|\chi_n\rangle\}$是很方便的
\begin{equation}\label{2}
  \hat{K}^\dag|\chi_n\rangle=\nu_n|\chi_n\rangle\quad\text{and}\quad\langle\chi_n|\hat{K}=\bar{\nu}_n\langle K_n|
\end{equation}
\end{document} 