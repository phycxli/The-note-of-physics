\documentclass{book}
\usepackage{amsmath}
\usepackage[UTF8]{ctex}
\usepackage{indentfirst}
\usepackage{enumerate}
\usepackage{graphicx}
\usepackage{wrapfig}
\usepackage{amssymb}
\usepackage{float}
\usepackage{subfigure}
\usepackage{cite}
\usepackage{caption}
\usepackage{setspace}
\usepackage{fancyhdr}
\usepackage{lastpage}
\usepackage{layout}
\usepackage{geometry}
\usepackage{tikz,mathpazo}
\usepackage[colorlinks,linkcolor=blue,anchorcolor=red,citecolor=green]{hyperref}

\begin{document}
\title{凝聚态场论}
\author{大白菜}
\date{2019-1-23}
\maketitle
\tableofcontents
\chapter{量子力学和基本量子场论}
\section{单体量子力学}
\section{多体量子力学}
\section{动力学原理和诺特原理}
\section{电磁场量子化}
\chapter{路径积分量子化}
\section{单粒子量子力学和路径积分}
\section{玻色子的路径积分}
\section{费米子的路径积分}
\section{规范场路径积分}
\section{自旋系统的路径积分}
\chapter{相变和对称性破缺}
\section{自发对称性破缺}
\section{Goldstone模式}
\section{KT相变}
\section{格点规范理论和紧闭问题}
\chapter{场论例子}
\section{RPA近似}
\section{超流的波戈留波夫理论}
\chapter{超导问题}
\section{超导和路径积分}
\section{约瑟夫结}
\section{二维量子涡旋中的超导——绝缘相变}
\chapter{量子霍尔液体和Chern-Simons规范场}
\section{二维电子系统}
\section{量子霍尔液体的有效理论}
\section{Laughlin波函数的推导}
\end{document}